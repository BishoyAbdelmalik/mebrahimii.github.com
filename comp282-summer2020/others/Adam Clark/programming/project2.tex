\documentclass{article}

\begin{document}

\title{COMP 282 - Project 2}
\author{}
\date{Fall 2018}
\maketitle

\section*{Instructions for Submission}

You must provide all source files in a single \texttt{.zip} file.  This file
must contain no directories whatsoever.  That is, if I were to unzip it, the
contents would be only the \texttt{.java} files required for the project.  For
example, in a *nix environment, I would create a submission by navigating to my
source directory issuing a \texttt{zip AdamClark *.java}.  {\bf Under no
circumstances will a submission be accepted if it is the product of you zipping
your IDE's project directory!}

There are to be absolutely no packages used in your submissions.  The use of
the default package is, generally, not advised for enterprise-level work; this,
however, is not that.

Additionally, you should not include any \texttt{main} functions in your
submission, only those files required for each project part are needed.

The following files {\bf must} be present in your submission:

\begin{itemize}
\item \texttt{Graph.java}
\end{itemize}

A series of ``checker'' functions have been provided for each part of the
project.  Ensure your project passes these basic checks prior to submission.
{\bf If any files are excluded, or fail to pass the checker, the grader {\em
will} fail, resulting in a 0 on the assignment.}

\section*{Part 1 - Undirected Graphs}

If you want to perform any further operations on graphs, you must first have a
data structure to operate on.  This section will concentrate on building such a
structure.

To begin with, we will start with what we know: that graphs consist of objects,
and their relationship to one another.  More formally, that a graph $G = (V,
E)$ where $V$ is the set of vertices (things) and $E$ is the set of edges
(their relationships to one another).

Just building this into a computer, however, is not straight-forward.  As
covered in lecture, taking the object-oriented approach here exhibits bad
locality of reference.  Therefore, we want to something a bit more
sophisticated than our previous work with trees.

\begin{verbatim}
import java.util.*;

public interface IGraph<T> {
  public ArrayList<T> getVertices();
  public ArrayList<ArrayList<Float>> getEdges();

  public void addVertex(T v);
  public void addEdge(T v1, T v2, float w);

  public void removeVertex(T v);
  public void removeEdge(T v1, T v2);
}
\end{verbatim}

The vertices should be placed into an array list in the order in which they
were added.  For example, if I added 4 vertices $v_0, v_1, v_2, v_3$ into an
array, they should be at indices 0, 1, 2, and 3, respectively.

Each of the \texttt{addVertex} and \texttt{addEdge} calls should behave in the
following manner.  When a new (unique) vertex is added, that vertex is assigned
a number (based on its index in the vertices array) that serves as an index
into the edges table.  Any new vertices added in this manner are completely
disconnected from the rest of the graph by default.  If an edge is added, it
obviously makes no sense to add it between one or two objects that do not exist
in the graph, in such cases, an error should be thrown.  {\bf The edges for a
graph are expected to be represented as an upper-triangular matrix!}

Similarly, trying to delete a vertex or an edge between two vertices with one
or more of them potentially not being part of the graph already, should result
in no operation being performed.  Only deletes on well-defined vertiecs and
edges are to be accepted.

{\bf You should not return the internal data structures as the result of a
\texttt{getVertices} or \texttt{getEdges} call!}  If you do this, it is
possible for someone to change their values without you knowing.  For example,
if I were to remove a vertex from the list of vertices you had returned, there
might still be a reference to it in the edges.  Instead, you {\bf must} clone
the internal class members in order to ensure this does not happen.

The checker for this project part is as follows:

\begin{verbatim}
import java.util.*;

public class GraphCheck {
  public static void run () {
    IGraph<Integer> gInts = new Graph<Integer>();
    IGraph<String> gStrs = new Graph<String>();
    IGraph<Float> gFlts = new Graph<Float>();

    gInts.addVertex(1);
    gStrs.addVertex("test");
    gFlts.addVertex(0.5f);

    gInts.addVertex(2);
    gInts.addEdge(1, 2, 1.0f);
    
    gStrs.addVertex("another");
    gStrs.addEdge("test", "another", 1.0f);

    gFlts.addVertex(2.0f);
    gFlts.addEdge(0.5f, 2.0f, 1.0f);

    ArrayList<Integer> vInts = gInts.getVertices();
    ArrayList<String> vStrs = gStrs.getVertices();
    ArrayList<Float> vFlts = gFlts.getVertices();

    ArrayList<ArrayList<Float>> eInts = gInts.getEdges();
  }
}
\end{verbatim}

\section*{Part 2 - Connected Components}

A connected component is simple a subset of vertices in a graph that are all
reachable from one another by following some path between them.  A graph is
considered connected if it contains one connected component -- consisting of
all the vertices in the graph.  In this section, we will used a graph traversal
to determine which connected component a vertex belongs in.

\begin{verbatim}
import java.util.*;

public interface IConnected<T> {
  ArrayList<T> componentOf(T v);
}
\end{verbatim}

This single-function interface requires that you start at a given vertex, and
find all other vertices in the graph connected to it.  That is, that you return
the connected component which includes the vertex passed to the function as an
argument.

If a vertex is provided to the \texttt{connectedTo} function that does not
exist in the graph, you must return an empty \texttt{ArrayList} to receive full
credit for this part of the assignment.

A simple checker for this part follows:

\begin{verbatim}
import java.util.*;

public class ConnectedCheck {
  public static void run () {
    IConnected<Integer> cInts = new Graph<Integer>();
    IConnected<String> cStrs = new Graph<String>();
    IConnected<Float> cFlts = new Graph<Float>();

    ArrayList<Integer> ints = cInts.componentOf(1);
    ArrayList<String> strs = cStrs.componentOf("test");
    ArrayList<Float> flts = cFlts.componentOf(3.0f);
  }
}
\end{verbatim}

Hint: You might want to consider implementing a private method,
\texttt{adjacentTo}, which will return all vertices adjacent (one edge away
from) the given vertex.

\end{document}
